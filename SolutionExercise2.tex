\documentclass[12pt,a4paper,twoside]{article}
%\usepackage{a4wide}
\usepackage{amsmath}
\usepackage{amssymb}
\usepackage[english]{babel}
\usepackage[all,cmtip]{xy}
\usepackage{graphicx}
\usepackage{pgfplots}
\pgfplotsset{compat=1.9}
\usepackage{tikz}
\usetikzlibrary{trees}
\usetikzlibrary{shapes}
\usepackage{enumitem}

\usepackage{fancyhdr}
\pagestyle{fancy}
\fancyhead[LO,RE]{Biostat I, week 2, April 2015}

\begin{document}

\section*{Solutions: Exercise 2}

\begin{enumerate}[label=\bfseries Q\arabic*.]
\item %Q1
  \begin{equation*}
    \left.\begin{aligned}
        FPR &= 0.01, \\
        FPR + TNR &= 1,\\
        \Pr(\text{All negative} \mid \text{no drug}) &= TNR_{1} \cdot TNR_{2} \cdot
        ...  \cdot TNR_{100}
      \end{aligned}
    \right\}
    0.99^{100} = 0.37
  \end{equation*}      
  

  $\Pr(\text{All negative} \mid \text{no drug}) = 0.37$.  Not so
  surprising ($\Pr=0.63$) that some tests would be positive.

\item %Q2
  \begin{align*}
    \text{Sensitivity (TPR)} = \frac{TP}{TP+FN} = \frac{80}{100} &= 0.80 \\
    \text{Specificity (TNR)} = \frac{TN}{TN+FP} = \frac{400-8}{400} &= 0.98 \\
    \text{False positive rate (FPR)} = \frac{FP}{TN+FP} = \frac{8}{400} &= 0.02 \\
    \text{False negative rate (FNR)} = \frac{FN}{TP+FN} = \frac{20}{100} &= 0.20 \\
  \end{align*}

\item %Q3
  \begin{align*}
    \text{Sensitivity (TPR)} = \frac{TP}{TP+FN} = \frac{133}{133+1} &\approx 0.99 \\
    \text{Specificity (TNR)} = \frac{TN}{TN+FP} = \frac{35}{35+12} &\approx 0.74 \\
    \text{Positive predictive value (PPV)} = \frac{TP}{TP+FP} = \frac{133}{133+12} &\approx 0.92 \\
    \text{Negative predictive value (NPV)} = \frac{TN}{TN+FN} = \frac{35}{35+1} &\approx 0.97 \\
  \end{align*}

\item %Q4
  \begin{enumerate}
  \item %Q4a
    \begin{equation*}
      \left.\begin{aligned}
          \Pr(FN) &= FNR,\\
          FNR + TPR &= 1
        \end{aligned}
      \right\}
      \qquad 1-0.85 = 0.15
    \end{equation*}      
    
  \item %Q4b
    \begin{equation*}
      \left.\begin{aligned}
          \Pr(FP) &= FPR,\\
          FPR + TNR &= 1
        \end{aligned}
      \right\}
      \qquad 1-0.80 = 0.20
    \end{equation*}      
    
  \item %Q4c
    % Define style for prob tree
    \tikzstyle{level 1}=[level distance=4cm, sibling distance=2cm]
    \tikzstyle{level 2}=[level distance=4.5cm, sibling distance=1cm]
    \tikzstyle{pop} = [text width=4em, text centered]
    \tikzstyle{disease} = [ellipse, minimum width=70pt, align=center, minimum height=30pt]
    \tikzstyle{dx} = [circle, minimum size=20pt]

    \begin{tikzpicture}[grow=right, sloped]
      \node[pop] {Random selection}
      child {
        node[disease, fill=green!20] {No Disease}
        child {
          node(TN)[dx, fill=green!20] {-}
          node[right of=TN] {TN}
          edge from parent
          node[above] {$0.80$}
        }
        child {
          node(FP)[dx, fill=red!20] {+}
          node[right of=FP] {FP}
          edge from parent
          node[above] {$0.20$}
        }
        edge from parent
        node[above] {$0.9975$}
      }
      child {
        node[disease, fill=red!20] {Disease}
        child {
          node(FN)[dx, fill=green!20] {-}
          node[right of=FN] {FN}
          edge from parent
          node[above] {$0.15$}
        }
        child {
          node(TP)[dx, fill=red!20] {+}
          node[right of=TP] {TP}
          edge from parent
          node[above] {$0.85$}
        }
        edge from parent
        node[above] {$0.0025$}
      };
    \end{tikzpicture}    


    \begin{align*}
      \Pr(+) &= \underbrace{0.0025 \cdot 0.85}_{TP}
               + \underbrace{0.9975 \cdot 0.20}_{FP},\\
      \Pr(\text{Disease} \cap +) &= \underbrace{0.0025 \cdot 0.85}_{TP},\\
      \Pr(\text{Disease} \mid +) &= PPV = \frac{\Pr(\text{Disease} \cap +)}{\Pr(+)} \approx 0.01
    \end{align*}
  \end{enumerate}

\item %Q5
  Well done if you solved this!
  \tikzstyle{level 1}=[level distance=4cm, sibling distance=2cm]
  \tikzstyle{level 2}=[level distance=4.5cm, sibling distance=1cm]
  \tikzstyle{pop} = [text width=4em, text centered]
  \tikzstyle{disease} = [ellipse, minimum width=70pt, align=center, minimum height=30pt]
  \tikzstyle{dx} = [circle, minimum size=20pt]
  
  \begin{tikzpicture}[grow=right, sloped]
    \node[pop] {Random selection}
    child {
      node[disease, fill=green!20] {No Disease}
      child {
        node(TN)[dx, fill=green!20] {-}
        node[right of=TN] {TN}
        edge from parent
        node[above] {$0.98$}
      }
      child {
        node(FP)[dx, fill=red!20] {+}
        node[right of=FP] {FP}
        edge from parent
        node[above] {$0.02$}
      }
      edge from parent
      node[above] {$1-p$}
    }
    child {
      node[disease, fill=red!20] {Disease}
      child {
        node(FN)[dx, fill=green!20] {-}
        node[right of=FN] {FN}
        edge from parent
        node[above] {$0.01$}
      }
      child {
        node(TP)[dx, fill=red!20] {+}
        node[right of=TP] {TP}
        edge from parent
        node[above] {$0.99$}
      }
      edge from parent
      node[above] {$p$}
    };
  \end{tikzpicture}    
  
  \begin{enumerate}
  \item Assume a prevalence ``p'', then: %Q5a
    \begin{align*}
      \Pr(+) &= \underbrace{p \cdot 0.99}_{TP} + 
               \underbrace{(1-p) \cdot 0.02}_{FP},\\
      \Pr(\text{Disease} \cap +) &= \underbrace{p \cdot 0.99}_{TP},\\
      \Pr(\text{Disease} \mid +) &= \frac{\Pr(\text{Disease} \cap+)}
                                   {\Pr(+)}
    \end{align*}
    Now can $\Pr(\text{Disease} \mid +) < 1\%$ ?
    \begin{align*}
      \Pr(\text{Disease} \mid +) = \frac{p \cdot 0.99}{p \cdot 0.99 + (1-p) \cdot 0.02} &< 0.01\\
      \frac{0.01 \cdot 0.02}{0.01 \cdot 0.02 + 0.99 - 0.01 \cdot
      0.99} \approx  \frac{204}{10^{6}} &< p
    \end{align*}
    So, yes, it is possible if the disease is more rare than 204 cases
    in a million.

  \item %Q5b   
    The implication is that for rare diseases a test needs to have very high validity.
  \end{enumerate}
  
\item %Q6

  \begin{enumerate}
    
  \item %Q6a
    Low sensitivity $\Rightarrow$ prevalence underestimated (if $FN > FP$).
    
  \item
    Low specificity $\Rightarrow$ prevalence may be overestimated
    (if $FP > FN$).	
    
    
    
    \tikzstyle{level 1}=[level distance=4.5cm, sibling distance=2cm]
    \tikzstyle{level 2}=[level distance=3.5cm, sibling distance=1cm]
    \tikzstyle{pop} = [text width=4em, text centered]
    \tikzstyle{disease} = [ellipse, minimum width=70pt, align=center, minimum height=30pt]
    \tikzstyle{dx} = [circle, minimum size=20pt]
    
    \begin{tikzpicture}[grow=right, sloped]
      \node[pop] {Random selection}
      child {
        node[disease, fill=green!20] {No Disease}
        child {
          node(TN)[dx, fill=green!20] {-}
          node[right of=TN] {TN}
          edge from parent
          node[above] {$0.98$}
        }
        child {
          node(FP)[dx, fill=red!20] {+}
          node[right of=FP] {FP}
          edge from parent
          node[above] {$0.02$}
        }
        edge from parent
        node[above] {$0.79 \ \& \ 0.93$}
      }
      child {
        node[disease, fill=red!20] {Disease}
        child {
          node(FN)[dx, fill=green!20] {-}
          node[right of=FN] {FN}
          edge from parent
          node[above] {$0.20$}
        }
        child {
          node(TP)[dx, fill=red!20] {+}
          node[right of=TP] {TP}
          edge from parent
          node[above] {$0.80$}
        }
        edge from parent
        node[above] {$0.21 \ \& \ 0.07$}
      };
    \end{tikzpicture}    

  \item By the diagnostic test reported prevalence:
    \begin{align*} %Why are we given the population size in the
      % question? Can we remove that from the question?
      \Pr(+)_{\text{First Pop}} = \underbrace{0.21 \cdot 0.80}_{TP} + 
      \underbrace{0.79 \cdot 0.02}_{FP} &\approx 0.18,\\
      \Pr(+)_{\text{Second Pop}} = \underbrace{0.07 \cdot 0.80}_{TP} + 
      \underbrace{0.93 \cdot 0.02}_{FP}
                                        &\approx 0.075                                          
    \end{align*}
    
  \item 
    The true rate ratio as stated in the question, and the rate ratio
    as reported by the diagnostic test:
    \begin{align*} 
      \text{Rate ratio}_{\text{ True}} &= \frac{0.21}{0.07} = 3,\\
      \text{Rate ratio}_{\text{ Diagnostic test}} &= \frac{0.18}{0.075} \approx 2.4                                          \end{align*}
  
  \end{enumerate}
\end{enumerate}
\end{document}